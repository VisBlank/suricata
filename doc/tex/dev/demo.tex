\chapter{Suricata 预研 Demo}\label{chap:demo}

\section{主要内容}

\begin{itemize}
    \item 环境搭建
    \item 协议分析(在前面的章节中,此处不表)
    \item 效果展示(IDS/IPS)
\end{itemize}

\section{网络结构}
Suricata 和数据库客户端均位于同一台机器。数据库服务器位于 182 以及 194。其网络结构如图所示:

\begin{tikzpicture}
    %\draw[help lines](0,0) grid(14,3);

    % 两个客户端工具
    \node [above] at (1.5, 1.75) {\cf mysql\_cli};
    \draw (0.5, 1.75) rectangle (2.5,2.25);
    \node [below] at (1.5, 1.25) {\cf oracle\_cli};
    \draw (0.5, 0.75) rectangle (2.5,1.25);

    % 宿主机
    \draw [rounded corners, purple, dashed](0,0) rectangle (7,3);
    \node [below] at (3.5, 0.5) {\cf 宿主机器(174)};

    % suricata
    \draw (4.5, 1) rectangle (6.5,2);
    \node at (5.5, 1.5) {\cf suricata};
    
    % db server
    \node [above] at (10.5, 1.75) {\cf mysql\_srv(182)};
    \draw (9, 1.75) rectangle (12,2.25);
    \node [below] at (10.5, 1.25) {\cf oracle\_srv(194)};
    \draw (9, 0.75) rectangle (12,1.25);

    % lines
    \draw[->,  teal] (1.5, 2.25) to [out=10,in=170] (10.5, 2.25);
    \draw[->,  teal] (1.5, 2.25) to [out=10, in=150] (5.5, 2);
    \draw[->,  teal] (1.5, 0.75) to [out=350,in=190] (10.5, 0.75);
    \draw[->,  teal] (1.5, 0.75) to [out=350, in=210] (5.5, 1);
    
    \draw[->,  blue] (2.5, 2) -- (4.5, 1.75);
    \draw[->,  blue] (6.5, 1.75) -- (9, 2);
    \draw[->,  blue] (2.5, 1) -- (4.5, 1.25);
    \draw[->,  blue] (6.5, 1.25) -- (9, 1);

    \draw[->, blue] (10,3.25) -- (11, 3.25);
    \node [right] at (11, 3.25) {\cf \scriptsize NFQ IPS 模式};
    \draw[->,  teal] (10,2.75) -- (11, 2.75);
    \node [right] at (11, 2.75) {\cf \scriptsize PCAP IDS 模式};
\end{tikzpicture}

\section{Suricata IDS/IPS 对现有数据库协议支持的展示}
目前对 MySQL 的协议以及基本支持,能获取对应的 SQL 语句以及更详细的信息。对 Oracle 11g 数据库而言,目前可以拿到其基本的 SQL 语句信息,其它更深入的信息(session 管理、返回的数据分析)尚不能获取。

\subsection{IDS 功能演示}

以命令 {\cf ./suricata r -c yaml/demo.yaml -i eth1} 启动 IDS 模式的 Suricata,此处使用 eth1 作为嗅探网卡。

对 Oracle 以及 MySQL 数据库而言,分别对应如下两条检测规则:

\begin{lstlisting}[language=python]
alert `\hl{red}{oracle11g}` any any -> any any (msg:"oracle user(coanor) detected";
    flow:to_server,established; `\hl{blue}{oracle11g-user}`:coanor; sid:2250000; rev:1;)
alert oracle11g any any -> any any (msg:"oracle database(orcl11g) detected";
    flow:to_server,established; `\hl{blue}{oracle11g-sid}`:orcl11g; sid:2250001; rev:1;)
\end{lstlisting}

\begin{lstlisting}[language=python]
alert `\hl{red}{mysql}` any any -> any any (msg:"mysql user(root) detected";
    flow:to_server,established; `\hl{blue}{mysql-user}`:root; sid:2240000; rev:1;)
alert mysql any any -> any any (msg:"mysql database(aap_log) detected";
    flow:to_server,established; `\hl{blue}{mysql-database}`:aap_log; sid:2240001; rev:1;)
\end{lstlisting}

对这两条规则而言,如果满足其检测条件,就会在 {\ff fast.log} 文件中记录相应的 alert 信息。同时,在 MySQL 以及 Oracle 的 JSON 文件日志中,也会记录这一行为(JSON 文件中的 {\cf action} 字段)。对其它几种行为而言,分别更改前面的 {\cf alert} 即可。

alert 日志在常规的 JSON 日志中为:

\begin{lstlisting}
{
    "timestamp":"2014-06-03T10:09:54.203021",
    "event_type":"oracle11g",
    "src_ip":"192.168.37.174",
    "src_port":45827,
    "dest_ip":"192.168.37.194",
    "dest_port":1521,
    "proto":"TCP",
    "oracle11g":{"user":"coanor",
        "db_name":"orcl11g",
        `\hl{blue}{"action":"ALERT"}`,
        "meta_info":{"sql":null, "cmd":"unkonw"}
    }
}
\end{lstlisting}

常规的 SQL 日志输出示例为:

\begin{lstlisting}
{
    "timestamp":"2014-06-03T10:12:30.534439",
    "event_type":"oracle11g",
    "src_ip":"192.168.37.174",
    "src_port":45827,
    "dest_ip":"192.168.37.194",
    "dest_port":1521,
    "proto":"TCP",
    "oracle11g":{
        "user":"coanor",
        "db_name":"orcl11g",
        "action":"UNKNOWN",
        "meta_info":{
            "sql":`\hl{blue}{"insert into employee (EMPLOYEE\_ID, MANAGER\_ID,
                   FIRST\_NAME, LAST\_NAME, TITLE, SALARY)
                   values( 1 , 0 , 'James' , 'Smith' , 'CEO', 800000)"}`,
            "cmd":"query"
        }
    }
}
\end{lstlisting}

在 {\ff fast.log} 文件中的示例为

\begin{lstlisting}
06/03/2014-13:38:41.827922  [**] [1:2250000:1] oracle user(coanor) detected [**]
    [Classification: (null)] [Priority: 3] {TCP} 192.168.37.174:46697 -> 192.168.37.194:1521
\end{lstlisting}

\subsection{IPS 功能演示}

设置如下 iptables 规则后,再启动 Suricata {\cf ./suricata -c yaml/demo.yaml -q 0}。

\begin{lstlisting}
sudo iptables -A INPUT -j NFQUEUE
sudo iptables -A OUTPUT -j NFQUEUE

# 允许 SSH 访问,便于调试
sudo iptables -I INPUT -p tcp -m tcp --dport 22 -j ACCEPT
sudo iptables -I OUTPUT -p tcp -m tcp --sport 22 -j ACCEPT
\end{lstlisting}

连接建立后,显示的结果数据和 IDS 模式下的一致。
